\noindent 
\begin{tabular}{|p{\lpmodwidth}|p{\lpnamewidth}|p{0.5in}|p{0.5in}|}
\cline{1-4}
\texttt{disksim\_disk} & \texttt{Model} & block & required \\ 
\cline{1-4}
\multicolumn{4}{|p{6in}|}{
Parameters for the disk's libdiskmodel model. See the diskmodel
documentation for details.
}\\ 
\cline{1-4}
\multicolumn{4}{p{5in}}{}\\
\end{tabular}\\ 
\noindent 
\begin{tabular}{|p{\lpmodwidth}|p{\lpnamewidth}|p{0.5in}|p{0.5in}|}
\cline{1-4}
\texttt{disksim\_disk} & \texttt{Scheduler} & block & required \\ 
\cline{1-4}
\multicolumn{4}{|p{6in}|}{
An ioqueue; see Section \ref{param.queue}
}\\ 
\cline{1-4}
\multicolumn{4}{p{5in}}{}\\
\end{tabular}\\ 
\noindent 
\begin{tabular}{|p{\lpmodwidth}|p{\lpnamewidth}|p{0.5in}|p{0.5in}|}
\cline{1-4}
\texttt{disksim\_disk} & \texttt{Max queue length} & int & required \\ 
\cline{1-4}
\multicolumn{4}{|p{6in}|}{
This specifies the maximum number of requests that the disk can have in
service or queued for service at any point in time. During
initialization, other components request this information and respect
it during simulation.
}\\ 
\cline{1-4}
\multicolumn{4}{p{5in}}{}\\
\end{tabular}\\ 
\noindent 
\begin{tabular}{|p{\lpmodwidth}|p{\lpnamewidth}|p{0.5in}|p{0.5in}|}
\cline{1-4}
\texttt{disksim\_disk} & \texttt{Bulk sector transfer time} & float & required \\ 
\cline{1-4}
\multicolumn{4}{|p{6in}|}{
This specifies the time for the disk to transfer a single 512-byte block
over the bus. Recall that this value is compared with the
corresponding bus and controller block transfer values to determine
the actual transfer time (i.e.,~the maximum of the three values).
}\\ 
\cline{1-4}
\multicolumn{4}{p{5in}}{}\\
\end{tabular}\\ 
\noindent 
\begin{tabular}{|p{\lpmodwidth}|p{\lpnamewidth}|p{0.5in}|p{0.5in}|}
\cline{1-4}
\texttt{disksim\_disk} & \texttt{Segment size (in blks)} & int & required \\ 
\cline{1-4}
\multicolumn{4}{|p{6in}|}{
This specifies the size of each buffer segment, assuming a static segment
size. Some modern disks will dynamically resize their buffer segments
(and thereby alter the number of segments) to respond to perceived
patterns of workload behavior, but DiskSim does not currently support
this functionality.
}\\ 
\cline{1-4}
\multicolumn{4}{p{5in}}{}\\
\end{tabular}\\ 
\noindent 
\begin{tabular}{|p{\lpmodwidth}|p{\lpnamewidth}|p{0.5in}|p{0.5in}|}
\cline{1-4}
\texttt{disksim\_disk} & \texttt{Number of buffer segments} & int & required \\ 
\cline{1-4}
\multicolumn{4}{|p{6in}|}{
This specifies the number of segments in the on-board buffer/cache. A
buffer segment is similar to a cache line, in that each segment
contains data that is disjoint from all other segments. However,
segments tend to be organized as circular queues of logically
sequential disk sectors, with new sectors pushed into an appropriate
queue either from the bus (during a write) or from the disk media
(during a read). As data are read from the buffer/cache and either
transferred over the bus (during a read) or written to the disk media
(during a write), they are eligible to be pushed out of the segment
(if necessary or according to the dictates of the buffer/cache
management algorithm).
}\\ 
\cline{1-4}
\multicolumn{4}{p{5in}}{}\\
\end{tabular}\\ 
\noindent 
\begin{tabular}{|p{\lpmodwidth}|p{\lpnamewidth}|p{0.5in}|p{0.5in}|}
\cline{1-4}
\texttt{disksim\_disk} & \texttt{Print stats} & int & required \\ 
\cline{1-4}
\multicolumn{4}{|p{6in}|}{
This specifies whether or not statistics for the disk will be reported.
}\\ 
\cline{1-4}
\multicolumn{4}{p{5in}}{}\\
\end{tabular}\\ 
\noindent 
\begin{tabular}{|p{\lpmodwidth}|p{\lpnamewidth}|p{0.5in}|p{0.5in}|}
\cline{1-4}
\texttt{disksim\_disk} & \texttt{Per-request overhead time} & float & required \\ 
\cline{1-4}
\multicolumn{4}{|p{6in}|}{
This specifies a per-request processing overhead that takes place
immediately after the arrival of a new request at the disk. It is
additive with various other processing overheads described below, but
in general either the other overheads are set to zero or this
parameter is set to zero.
}\\ 
\cline{1-4}
\multicolumn{4}{p{5in}}{}\\
\end{tabular}\\ 
\noindent 
\begin{tabular}{|p{\lpmodwidth}|p{\lpnamewidth}|p{0.5in}|p{0.5in}|}
\cline{1-4}
\texttt{disksim\_disk} & \texttt{Time scale for overheads} & float & required \\ 
\cline{1-4}
\multicolumn{4}{|p{6in}|}{
This specifies a multiplicative scaling factor for all processing overhead
times. For example, 0.0 eliminates all such delays, 1.0 uses them at
face value, and 1.5 increases them all by 50\%.
}\\ 
\cline{1-4}
\multicolumn{4}{p{5in}}{}\\
\end{tabular}\\ 
\noindent 
\begin{tabular}{|p{\lpmodwidth}|p{\lpnamewidth}|p{0.5in}|p{0.5in}|}
\cline{1-4}
\texttt{disksim\_disk} & \texttt{Hold bus entire read xfer} & int & required \\ 
\cline{1-4}
\multicolumn{4}{|p{6in}|}{
This specifies whether or not the disk retains ownership of the bus
throughout the entire transfer of ``read'' data from the disk. If
false~(0), the disk may release the bus if and when the current
transfer has exhausted all of the available data in the on-board
buffer/cache and must wait for additional data sectors to be read off
the disk into the buffer/cache.
}\\ 
\cline{1-4}
\multicolumn{4}{p{5in}}{}\\
\end{tabular}\\ 
\noindent 
\begin{tabular}{|p{\lpmodwidth}|p{\lpnamewidth}|p{0.5in}|p{0.5in}|}
\cline{1-4}
\texttt{disksim\_disk} & \texttt{Hold bus entire write xfer} & int & required \\ 
\cline{1-4}
\multicolumn{4}{|p{6in}|}{
This specifies whether or not the disk retains ownership of the bus
throughout the entire transfer of ``write'' data to the disk. If
false~(0), the disk may release the bus if and when the current
transfer has filled up the available space in the on-board
buffer/cache and must wait for data from the buffer/cache to be
written to the disk.
}\\ 
\cline{1-4}
\multicolumn{4}{p{5in}}{}\\
\end{tabular}\\ 
\noindent 
\begin{tabular}{|p{\lpmodwidth}|p{\lpnamewidth}|p{0.5in}|p{0.5in}|}
\cline{1-4}
\texttt{disksim\_disk} & \texttt{Allow almost read hits} & int & required \\ 
\cline{1-4}
\multicolumn{4}{|p{6in}|}{
This specifies whether or not a new read request whose first block is
currently being prefetched should be treated as a partial cache hit.
Doing so generally means that the request is handled right away.
}\\ 
\cline{1-4}
\multicolumn{4}{p{5in}}{}\\
\end{tabular}\\ 
\noindent 
\begin{tabular}{|p{\lpmodwidth}|p{\lpnamewidth}|p{0.5in}|p{0.5in}|}
\cline{1-4}
\texttt{disksim\_disk} & \texttt{Allow sneaky full read hits} & int & required \\ 
\cline{1-4}
\multicolumn{4}{|p{6in}|}{
This specifies whether or not a new read request whose data are entirely
contained in a single segment of the disk cache is allowed to
immediately transfer that data over the bus while another request is
moving the disk actuator and/or transferring data between the disk
cache and the disk media. In essence, the new read request ``sneaks''
its data out from the disk cache without interrupting the current
(active) disk request.
}\\ 
\cline{1-4}
\multicolumn{4}{p{5in}}{}\\
\end{tabular}\\ 
\noindent 
\begin{tabular}{|p{\lpmodwidth}|p{\lpnamewidth}|p{0.5in}|p{0.5in}|}
\cline{1-4}
\texttt{disksim\_disk} & \texttt{Allow sneaky partial read hits} & int & required \\ 
\cline{1-4}
\multicolumn{4}{|p{6in}|}{
This specifies whether or not a new read request whose data are partially
contained in a single segment of the disk cache is allowed to
immediately transfer that data over the bus while another request is
moving the disk actuator and/or transferring data between the disk
cache and the disk media. In essence, the new read request ``sneaks''
the cached portion of its data out from the disk cache without
interrupting the current (active) disk request.
}\\ 
\cline{1-4}
\multicolumn{4}{p{5in}}{}\\
\end{tabular}\\ 
\noindent 
\begin{tabular}{|p{\lpmodwidth}|p{\lpnamewidth}|p{0.5in}|p{0.5in}|}
\cline{1-4}
\texttt{disksim\_disk} & \texttt{Allow sneaky intermediate read hits} & int & required \\ 
\cline{1-4}
\multicolumn{4}{|p{6in}|}{
This specifies whether or not the on-board queue of requests is searched
during idle bus periods in order to find read requests that may be
partially or completely serviced from the current contents of the disk
cache. That is, if the current (active) request does not need bus
access at the current time, and the bus is available for use, a queued
read request whose data are in the cache may obtain access to the bus
and begin data transfer. ``Full'' intermediate read hits are given
precedence over ``partial'' intermediate read hits.
}\\ 
\cline{1-4}
\multicolumn{4}{p{5in}}{}\\
\end{tabular}\\ 
\noindent 
\begin{tabular}{|p{\lpmodwidth}|p{\lpnamewidth}|p{0.5in}|p{0.5in}|}
\cline{1-4}
\texttt{disksim\_disk} & \texttt{Allow read hits on write data} & int & required \\ 
\cline{1-4}
\multicolumn{4}{|p{6in}|}{
This specifies whether or not data placed in the disk cache by write
requests are considered usable by read requests. If false~(0), such
data are removed from the cache as soon as they have been copied to
the media.
}\\ 
\cline{1-4}
\multicolumn{4}{p{5in}}{}\\
\end{tabular}\\ 
\noindent 
\begin{tabular}{|p{\lpmodwidth}|p{\lpnamewidth}|p{0.5in}|p{0.5in}|}
\cline{1-4}
\texttt{disksim\_disk} & \texttt{Allow write prebuffering} & int & required \\ 
\cline{1-4}
\multicolumn{4}{|p{6in}|}{
This specifies whether or not the on-board queue of requests is searched
during idle bus periods for write requests that could have part or all
of their data transferred to the on-board cache (without disturbing an
ongoing request). That is, if the current (active) request does not
need bus access at the current time, and the bus is available for use,
a queued write request may obtain access to the bus and begin data
transfer into an appropriate cache segment. Writes that are
contiguous to the end of the current (active) request are given
highest priority in order to facilitate continuous transfer to the
media, followed by writes that have already ``prebuffered'' some
portion of their data.
}\\ 
\cline{1-4}
\multicolumn{4}{p{5in}}{}\\
\end{tabular}\\ 
\noindent 
\begin{tabular}{|p{\lpmodwidth}|p{\lpnamewidth}|p{0.5in}|p{0.5in}|}
\cline{1-4}
\texttt{disksim\_disk} & \texttt{Preseeking level} & int & required \\ 
\cline{1-4}
\multicolumn{4}{|p{6in}|}{
This specifies how soon the actuator is allowed to start seeking towards
the media location of the next request's data.
0~indicates no preseeking, meaning that the actuator does not begin
relocation until the current request's completion has been confirmed
by the controller (via a completion ``handshake'' over the bus).
1~indicates that the actuator can begin relocation as soon as the
completion message has been prepared for transmission by the disk.
2~indicates that the actuator can begin relocation as soon as the
access of the last sector of the current request (and any required
prefetching) has been completed. This allows greater parallelism
between bus activity and mechanical activity.
}\\ 
\cline{1-4}
\multicolumn{4}{p{5in}}{}\\
\end{tabular}\\ 
\noindent 
\begin{tabular}{|p{\lpmodwidth}|p{\lpnamewidth}|p{0.5in}|p{0.5in}|}
\cline{1-4}
\texttt{disksim\_disk} & \texttt{Never disconnect} & int & required \\ 
\cline{1-4}
\multicolumn{4}{|p{6in}|}{
This specifies whether or not the disk retains ownership of the bus during
the entire processing and servicing of a request (i.e.,~from arrival
to completion). If false~(0), the disk may release the bus whenever
it is not needed for transferring data or control information.
}\\ 
\cline{1-4}
\multicolumn{4}{p{5in}}{}\\
\end{tabular}\\ 
\noindent 
\begin{tabular}{|p{\lpmodwidth}|p{\lpnamewidth}|p{0.5in}|p{0.5in}|}
\cline{1-4}
\texttt{disksim\_disk} & \texttt{Avg sectors per cylinder} & int & required \\ 
\cline{1-4}
\multicolumn{4}{|p{6in}|}{
This specifies (an approximation of) the mean number of sectors per
cylinder. This value is exported to external schedulers some of which
utilize approximations of the actual layout of data on the disk(s)
when reordering disk requests. This value is not used by the disk
itself.
}\\ 
\cline{1-4}
\multicolumn{4}{p{5in}}{}\\
\end{tabular}\\ 
\noindent 
\begin{tabular}{|p{\lpmodwidth}|p{\lpnamewidth}|p{0.5in}|p{0.5in}|}
\cline{1-4}
\texttt{disksim\_disk} & \texttt{Maximum number of write segments} & int & required \\ 
\cline{1-4}
\multicolumn{4}{|p{6in}|}{
This specifies the number of cache segments available for holding
``write'' data at any point in time. Because write-back caching is
typically quite limited in current disk cache management schemes, some
caches only allow a subset of the segments to be used to hold data for
write requests (in order to minimize any interference with sequential
read streams).
}\\ 
\cline{1-4}
\multicolumn{4}{p{5in}}{}\\
\end{tabular}\\ 
\noindent 
\begin{tabular}{|p{\lpmodwidth}|p{\lpnamewidth}|p{0.5in}|p{0.5in}|}
\cline{1-4}
\texttt{disksim\_disk} & \texttt{Use separate write segment} & int & required \\ 
\cline{1-4}
\multicolumn{4}{|p{6in}|}{
This specifies whether or not a single segment is statically designated
for use by all write requests. This further minimizes the impact of
write requests on the caching/prefetching of sequential read streams.
}\\ 
\cline{1-4}
\multicolumn{4}{p{5in}}{}\\
\end{tabular}\\ 
\noindent 
\begin{tabular}{|p{\lpmodwidth}|p{\lpnamewidth}|p{0.5in}|p{0.5in}|}
\cline{1-4}
\texttt{disksim\_disk} & \texttt{Low (write) water mark} & float & required \\ 
\cline{1-4}
\multicolumn{4}{|p{6in}|}{
This specifies the fraction of segment size or request size (see below)
corresponding to the {\it low water mark}. When data for a write
request are being transferred over the bus into the buffer/cache, and
the buffer/cache segment fills up with ``dirty'' data, the disk may
disconnect from the bus while the buffered data are written to the
disk media. When the amount of dirty data in the cache falls below
the low water mark, the disk attempts to reconnect to the bus to
continue the interrupted data transfer.
}\\ 
\cline{1-4}
\multicolumn{4}{p{5in}}{}\\
\end{tabular}\\ 
\noindent 
\begin{tabular}{|p{\lpmodwidth}|p{\lpnamewidth}|p{0.5in}|p{0.5in}|}
\cline{1-4}
\texttt{disksim\_disk} & \texttt{High (read) water mark} & float & required \\ 
\cline{1-4}
\multicolumn{4}{|p{6in}|}{
This specifies the fraction of
segment size or request size (see below) corresponding to the
{\it high water mark}. When data for a read request are being transferred
over the bus from the buffer/cache, and the buffer/cache segment runs out of
data to transfer, the disk may disconnect from the bus until additional data
are read from the disk media. When the amount of available data in the cache
reaches the high water mark, the disk attempts to reconnect to the bus to
continue the interrupted data transfer.
}\\ 
\cline{1-4}
\multicolumn{4}{p{5in}}{}\\
\end{tabular}\\ 
\noindent 
\begin{tabular}{|p{\lpmodwidth}|p{\lpnamewidth}|p{0.5in}|p{0.5in}|}
\cline{1-4}
\texttt{disksim\_disk} & \texttt{Set watermark by reqsize} & int & required \\ 
\cline{1-4}
\multicolumn{4}{|p{6in}|}{
This specifies whether the
watermarks are computed as fractions of the individual request size
or as fractions of the buffer/cache segment size.
}\\ 
\cline{1-4}
\multicolumn{4}{p{5in}}{}\\
\end{tabular}\\ 
\noindent 
\begin{tabular}{|p{\lpmodwidth}|p{\lpnamewidth}|p{0.5in}|p{0.5in}|}
\cline{1-4}
\texttt{disksim\_disk} & \texttt{Calc sector by sector} & int & required \\ 
\cline{1-4}
\multicolumn{4}{|p{6in}|}{
This specifies whether or not media transfers should be computed sector by
sector rather than in groups of sectors. This optimization has no
effect on simulation accuracy, but potentially results in shorter
simulation times (at a cost of increased code complexity). It has not
been re-enabled since the most recent modifications to DiskSim, so the
simulator currently functions as if the value were always true~(1).
}\\ 
\cline{1-4}
\multicolumn{4}{p{5in}}{}\\
\end{tabular}\\ 
\noindent 
\begin{tabular}{|p{\lpmodwidth}|p{\lpnamewidth}|p{0.5in}|p{0.5in}|}
\cline{1-4}
\texttt{disksim\_disk} & \texttt{Enable caching in buffer} & int & required \\ 
\cline{1-4}
\multicolumn{4}{|p{6in}|}{
This specifies whether or not
on-board buffer segments are used for data caching as well as for
speed-matching between the bus and the disk media. Most (if not all) modern
disk drives utilize their buffers as caches.
}\\ 
\cline{1-4}
\multicolumn{4}{p{5in}}{}\\
\end{tabular}\\ 
\noindent 
\begin{tabular}{|p{\lpmodwidth}|p{\lpnamewidth}|p{0.5in}|p{0.5in}|}
\cline{1-4}
\texttt{disksim\_disk} & \texttt{Buffer continuous read} & int & required \\ 
\cline{1-4}
\multicolumn{4}{|p{6in}|}{
This specifies the type of prefetching performed by the disk. 0~disables
prefetching. 1~enables prefetching up to the end of the track
containing the last sector of the read request. 2~enables prefetching
up to the end of the cylinder containing the last sector of the read
request. 3~enables prefetching up to the point that the current cache
segment is full. 4~enables prefetching up to the end of the track
following the track containing the last sector of the read request,
provided that the current request was preceded in the not-too-distant
past by another read request that accessed the immediately previous
track. In essence, the last scheme enables a type of prefetching that
tries to stay one logical track ``ahead'' of any sequential read
streams that are detected.
}\\ 
\cline{1-4}
\multicolumn{4}{p{5in}}{}\\
\end{tabular}\\ 
\noindent 
\begin{tabular}{|p{\lpmodwidth}|p{\lpnamewidth}|p{0.5in}|p{0.5in}|}
\cline{1-4}
\texttt{disksim\_disk} & \texttt{Minimum read-ahead (blks)} & int & required \\ 
\cline{1-4}
\multicolumn{4}{|p{6in}|}{
This specifies the minimum number of disk sectors that must be prefetched
after a read request before servicing another (read or write) request.
A positive value may be beneficial for workloads containing multiple
interleaved sequential read streams, but 0~is typically the
appropriate value.
}\\ 
\cline{1-4}
\multicolumn{4}{p{5in}}{}\\
\end{tabular}\\ 
\noindent 
\begin{tabular}{|p{\lpmodwidth}|p{\lpnamewidth}|p{0.5in}|p{0.5in}|}
\cline{1-4}
\texttt{disksim\_disk} & \texttt{Maximum read-ahead (blks)} & int & required \\ 
\cline{1-4}
\multicolumn{4}{|p{6in}|}{
This specifies the maximum number of disk sectors that may be prefetched
after a read request (regardless of all other prefetch parameters).
}\\ 
\cline{1-4}
\multicolumn{4}{p{5in}}{}\\
\end{tabular}\\ 
\noindent 
\begin{tabular}{|p{\lpmodwidth}|p{\lpnamewidth}|p{0.5in}|p{0.5in}|}
\cline{1-4}
\texttt{disksim\_disk} & \texttt{Read-ahead over requested} & int & required \\ 
\cline{1-4}
\multicolumn{4}{|p{6in}|}{
This specifies whether or not newly prefetched data can replace (in a
buffer segment) data returned to the host as part of the most recent
read request.
% If false~(0), prefetch is %terminated when the buffer segment is
% full.
}\\ 
\cline{1-4}
\multicolumn{4}{p{5in}}{}\\
\end{tabular}\\ 
\noindent 
\begin{tabular}{|p{\lpmodwidth}|p{\lpnamewidth}|p{0.5in}|p{0.5in}|}
\cline{1-4}
\texttt{disksim\_disk} & \texttt{Read-ahead on idle hit} & int & required \\ 
\cline{1-4}
\multicolumn{4}{|p{6in}|}{
This specifies whether or not prefetching should be initiated by the
disk when a read request is completely satisfied by cached data
(i.e.,~a ``full read hit'').
}\\ 
\cline{1-4}
\multicolumn{4}{p{5in}}{}\\
\end{tabular}\\ 
\noindent 
\begin{tabular}{|p{\lpmodwidth}|p{\lpnamewidth}|p{0.5in}|p{0.5in}|}
\cline{1-4}
\texttt{disksim\_disk} & \texttt{Read any free blocks} & int & required \\ 
\cline{1-4}
\multicolumn{4}{|p{6in}|}{
This specifies whether or not disk sectors logically prior to the
requested sectors should be read into the cache if they pass under the
read/write head prior to reaching the requested data (e.g.,~during
rotational latency).
}\\ 
\cline{1-4}
\multicolumn{4}{p{5in}}{}\\
\end{tabular}\\ 
\noindent 
\begin{tabular}{|p{\lpmodwidth}|p{\lpnamewidth}|p{0.5in}|p{0.5in}|}
\cline{1-4}
\texttt{disksim\_disk} & \texttt{Fast write level} & int & required \\ 
\cline{1-4}
\multicolumn{4}{|p{6in}|}{
This specifies the type of write-back caching implemented.
0~indicates that write-back caching is disabled (i.e.,~all dirty data
must be written to the disk media prior to sending a completion
message).
1~indicates that write-back caching is enabled for contiguous
sequential write request streams. That is, as long as each request
arriving at the disk is a write request that ``appends'' to the
current segment of dirty data, a completion message will be returned
for each new request as soon as all of its data have been transferred
over the bus to the disk buffer/cache.
2~indicates that write-back caching is enabled for contiguous
sequential write request streams even if they are intermixed with read
or non-appending write requests, although before any such request is
serviced by the disk, all of the dirty write data must be flushed to
the media. A scheduling algorithm that gives precedence to sequential
writes would maximize the effectiveness of this option.
}\\ 
\cline{1-4}
\multicolumn{4}{p{5in}}{}\\
\end{tabular}\\ 
\noindent 
\begin{tabular}{|p{\lpmodwidth}|p{\lpnamewidth}|p{0.5in}|p{0.5in}|}
\cline{1-4}
\texttt{disksim\_disk} & \texttt{Combine seq writes} & int & required \\ 
\cline{1-4}
\multicolumn{4}{|p{6in}|}{
This specifies whether or not sequential data from separate write requests
can share a common cache segment. If true~(1), data are typically
appended at the end of a previous request's dirty data. However, if
all of the data in a cache segment are dirty, and no mechanical
activity has begun on behalf of the request(s) using that segment,
``prepending'' of additional dirty data are allowed provided that the
resulting cache segment contains a single contiguous set of dirty
sectors.
}\\ 
\cline{1-4}
\multicolumn{4}{p{5in}}{}\\
\end{tabular}\\ 
\noindent 
\begin{tabular}{|p{\lpmodwidth}|p{\lpnamewidth}|p{0.5in}|p{0.5in}|}
\cline{1-4}
\texttt{disksim\_disk} & \texttt{Stop prefetch in sector} & int & required \\ 
\cline{1-4}
\multicolumn{4}{|p{6in}|}{
This specifies whether or not a prefetch may be aborted in the ``middle''
of reading a sector off the media. If false~(0), prefetch activity is
only aborted at sector boundaries.
}\\ 
\cline{1-4}
\multicolumn{4}{p{5in}}{}\\
\end{tabular}\\ 
\noindent 
\begin{tabular}{|p{\lpmodwidth}|p{\lpnamewidth}|p{0.5in}|p{0.5in}|}
\cline{1-4}
\texttt{disksim\_disk} & \texttt{Disconnect write if seek} & int & required \\ 
\cline{1-4}
\multicolumn{4}{|p{6in}|}{
This specifies whether or not the disk should disconnect from the bus if
the actuator is still in motion (seeking) when the last of a write
request's data has been transferred to the disk buffer/cache.
}\\ 
\cline{1-4}
\multicolumn{4}{p{5in}}{}\\
\end{tabular}\\ 
\noindent 
\begin{tabular}{|p{\lpmodwidth}|p{\lpnamewidth}|p{0.5in}|p{0.5in}|}
\cline{1-4}
\texttt{disksim\_disk} & \texttt{Write hit stop prefetch} & int & required \\ 
\cline{1-4}
\multicolumn{4}{|p{6in}|}{
This specifies whether or not
the disk should discontinue the read-ahead of a previous request when a
write hit in the cache occurs. Doing so allows the new write request's
data to begin travelling to the disk more quickly, at the expense of
some prefetching activity.
}\\ 
\cline{1-4}
\multicolumn{4}{p{5in}}{}\\
\end{tabular}\\ 
\noindent 
\begin{tabular}{|p{\lpmodwidth}|p{\lpnamewidth}|p{0.5in}|p{0.5in}|}
\cline{1-4}
\texttt{disksim\_disk} & \texttt{Read directly to buffer} & int & required \\ 
\cline{1-4}
\multicolumn{4}{|p{6in}|}{
This specifies whether or not
space for a sector must be available in the buffer/cache prior to the
start of the sector read. If false~(0), a separate sector buffer is assumed
to be available for use by the media-reading electronics, implying that the
data for a sector is transferred to the main buffer/cache only after it has
been completely read (and any error-correction algorithms completed).
}\\ 
\cline{1-4}
\multicolumn{4}{p{5in}}{}\\
\end{tabular}\\ 
\noindent 
\begin{tabular}{|p{\lpmodwidth}|p{\lpnamewidth}|p{0.5in}|p{0.5in}|}
\cline{1-4}
\texttt{disksim\_disk} & \texttt{Immed transfer partial hit} & int & required \\ 
\cline{1-4}
\multicolumn{4}{|p{6in}|}{
This specifies whether or not a read request whose initial (but not all)
data are present in the disk buffer/cache has that data immediately
transferred over the bus. If false~(0), the data are immediately
transferred only if the amount of requested data that are present in
the buffer/cache exceed the {\it high water mark} (see above).
}\\ 
\cline{1-4}
\multicolumn{4}{p{5in}}{}\\
\end{tabular}\\ 
\noindent 
\begin{tabular}{|p{\lpmodwidth}|p{\lpnamewidth}|p{0.5in}|p{0.5in}|}
\cline{1-4}
\texttt{disksim\_disk} & \texttt{Read hit over. after read} & float & required \\ 
\cline{1-4}
\multicolumn{4}{|p{6in}|}{
This specifies the processing time for a read request that hits in the
on-board cache when the immediately previous request was also a read.
This delay is applied before any data are transferred from the disk
buffer/cache.
}\\ 
\cline{1-4}
\multicolumn{4}{p{5in}}{}\\
\end{tabular}\\ 
\noindent 
\begin{tabular}{|p{\lpmodwidth}|p{\lpnamewidth}|p{0.5in}|p{0.5in}|}
\cline{1-4}
\texttt{disksim\_disk} & \texttt{Read hit over. after write} & float & required \\ 
\cline{1-4}
\multicolumn{4}{|p{6in}|}{
This specifies the processing time for a read request that hits in the
on-board cache when the immediately previous request was a write.
This delay is applied before any data are transferred from the disk
buffer/cache.
}\\ 
\cline{1-4}
\multicolumn{4}{p{5in}}{}\\
\end{tabular}\\ 
\noindent 
\begin{tabular}{|p{\lpmodwidth}|p{\lpnamewidth}|p{0.5in}|p{0.5in}|}
\cline{1-4}
\texttt{disksim\_disk} & \texttt{Read miss over. after read} & float & required \\ 
\cline{1-4}
\multicolumn{4}{|p{6in}|}{
This specifies the processing time for a read request that misses in the
on-board cache when the immediately previous request was also a read.
This delay is applied before any mechanical positioning delays or data
transfer from the media.
}\\ 
\cline{1-4}
\multicolumn{4}{p{5in}}{}\\
\end{tabular}\\ 
\noindent 
\begin{tabular}{|p{\lpmodwidth}|p{\lpnamewidth}|p{0.5in}|p{0.5in}|}
\cline{1-4}
\texttt{disksim\_disk} & \texttt{Read miss over. after write} & float & required \\ 
\cline{1-4}
\multicolumn{4}{|p{6in}|}{
This specifies the processing time for a read request that misses in the
on-board cache when the immediately previous request was a write.
This delay is applied before any mechanical positioning delays or data
transfer from the media.
}\\ 
\cline{1-4}
\multicolumn{4}{p{5in}}{}\\
\end{tabular}\\ 
\noindent 
\begin{tabular}{|p{\lpmodwidth}|p{\lpnamewidth}|p{0.5in}|p{0.5in}|}
\cline{1-4}
\texttt{disksim\_disk} & \texttt{Write hit over. after read} & float & required \\ 
\cline{1-4}
\multicolumn{4}{|p{6in}|}{
This specifies the processing time for a write request that ``hits'' in
the on-board cache (i.e., completion will be reported before data
reaches media) when the immediately previous request was a read. This
delay is applied before any mechanical positioning delays and before
any data are transferred to the disk buffer/cache.
}\\ 
\cline{1-4}
\multicolumn{4}{p{5in}}{}\\
\end{tabular}\\ 
\noindent 
\begin{tabular}{|p{\lpmodwidth}|p{\lpnamewidth}|p{0.5in}|p{0.5in}|}
\cline{1-4}
\texttt{disksim\_disk} & \texttt{Write hit over. after write} & float & required \\ 
\cline{1-4}
\multicolumn{4}{|p{6in}|}{
This specifies the processing time for a write request that ``hits'' in
the on-board cache (i.e., completion will be reported before data
reaches media) when the immediately previous request was also a write.
This delay is applied before any mechanical positioning delays and
before any data are transferred to the disk buffer/cache.
}\\ 
\cline{1-4}
\multicolumn{4}{p{5in}}{}\\
\end{tabular}\\ 
\noindent 
\begin{tabular}{|p{\lpmodwidth}|p{\lpnamewidth}|p{0.5in}|p{0.5in}|}
\cline{1-4}
\texttt{disksim\_disk} & \texttt{Write miss over. after read} & float & required \\ 
\cline{1-4}
\multicolumn{4}{|p{6in}|}{
This specifies that ``misses'' in the on-board cache (i.e., completion will
be reported only after data reaches media) when the immediately
previous request was a read. This delay is applied before any
mechanical positioning delays and before any data are transferred to
the disk buffer/cache.
}\\ 
\cline{1-4}
\multicolumn{4}{p{5in}}{}\\
\end{tabular}\\ 
\noindent 
\begin{tabular}{|p{\lpmodwidth}|p{\lpnamewidth}|p{0.5in}|p{0.5in}|}
\cline{1-4}
\texttt{disksim\_disk} & \texttt{Write miss over. after write} & float & required \\ 
\cline{1-4}
\multicolumn{4}{|p{6in}|}{
This specifies the processing time for a write request that ``misses'' in
the on-board cache (i.e., completion will be reported only after data
reaches media) when the immediately previous request was also a write.
This delay is applied before any mechanical positioning delays and
before any data are transferred to the disk buffer/cache.
}\\ 
\cline{1-4}
\multicolumn{4}{p{5in}}{}\\
\end{tabular}\\ 
\noindent 
\begin{tabular}{|p{\lpmodwidth}|p{\lpnamewidth}|p{0.5in}|p{0.5in}|}
\cline{1-4}
\texttt{disksim\_disk} & \texttt{Read completion overhead} & float & required \\ 
\cline{1-4}
\multicolumn{4}{|p{6in}|}{
This specifies the processing time for completing a read request.
This overhead is applied just before the completion message is sent
over the (previously acquired) bus and occurs in parallel with any
background disk activity (e.g.,~prefetching or preseeking).
}\\ 
\cline{1-4}
\multicolumn{4}{p{5in}}{}\\
\end{tabular}\\ 
\noindent 
\begin{tabular}{|p{\lpmodwidth}|p{\lpnamewidth}|p{0.5in}|p{0.5in}|}
\cline{1-4}
\texttt{disksim\_disk} & \texttt{Write completion overhead} & float & required \\ 
\cline{1-4}
\multicolumn{4}{|p{6in}|}{
This specifies the processing time for completing a write request.
This overhead is applied just before the completion message is sent
over the (previously acquired) bus and occurs in parallel with any
background disk activity (e.g.,~preseeking).
}\\ 
\cline{1-4}
\multicolumn{4}{p{5in}}{}\\
\end{tabular}\\ 
\noindent 
\begin{tabular}{|p{\lpmodwidth}|p{\lpnamewidth}|p{0.5in}|p{0.5in}|}
\cline{1-4}
\texttt{disksim\_disk} & \texttt{Data preparation overhead} & float & required \\ 
\cline{1-4}
\multicolumn{4}{|p{6in}|}{
This specifies the additional processing time necessary when preparing to
transfer data over the bus (for either reads or writes). This command
processing overhead is applied after obtaining access to the bus
(prior to transferring any data) and occurs in parallel with any
ongoing media access.
}\\ 
\cline{1-4}
\multicolumn{4}{p{5in}}{}\\
\end{tabular}\\ 
\noindent 
\begin{tabular}{|p{\lpmodwidth}|p{\lpnamewidth}|p{0.5in}|p{0.5in}|}
\cline{1-4}
\texttt{disksim\_disk} & \texttt{First reselect overhead} & float & required \\ 
\cline{1-4}
\multicolumn{4}{|p{6in}|}{
This specifies the processing time for reconnecting to (or
``reselecting'') the controller for the first time during the current
request. This command processing overhead is applied after the disk
determines that reselection is appropriate (prior to attempting to
acquire the bus) and occurs in parallel with any ongoing media access.
Reselection implies that the disk has explicitly disconnected from the
bus at some previous point while servicing the current request and is
now attempting to reestablish communication with the controller.
Disconnection and subsequent reselection result in some additional
command processing and protocol overhead, but they may also improve
the overall utilization of bus resources shared by multiple disks (or
other peripherals).
}\\ 
\cline{1-4}
\multicolumn{4}{p{5in}}{}\\
\end{tabular}\\ 
\noindent 
\begin{tabular}{|p{\lpmodwidth}|p{\lpnamewidth}|p{0.5in}|p{0.5in}|}
\cline{1-4}
\texttt{disksim\_disk} & \texttt{Other reselect overhead} & float & required \\ 
\cline{1-4}
\multicolumn{4}{|p{6in}|}{
This specifies the processing time for reconnecting to the controller
after the first time during the current request (i.e.,~the second
reselection, the third reselection, etc.). This command processing
overhead is applied after the disk determines that reselection is
appropriate (prior to attempting to acquire the bus) and occurs in
parallel with any ongoing media access.
}\\ 
\cline{1-4}
\multicolumn{4}{p{5in}}{}\\
\end{tabular}\\ 
\noindent 
\begin{tabular}{|p{\lpmodwidth}|p{\lpnamewidth}|p{0.5in}|p{0.5in}|}
\cline{1-4}
\texttt{disksim\_disk} & \texttt{Read disconnect afterread} & float & required \\ 
\cline{1-4}
\multicolumn{4}{|p{6in}|}{
This specifies the processing time for a read request that disconnects
from the bus when the previous request was also a read. This command
processing overhead is applied after the disk determines that
disconnection is appropriate (prior to requesting disconnection from
the bus) and occurs in parallel with any ongoing media access.
}\\ 
\cline{1-4}
\multicolumn{4}{p{5in}}{}\\
\end{tabular}\\ 
\noindent 
\begin{tabular}{|p{\lpmodwidth}|p{\lpnamewidth}|p{0.5in}|p{0.5in}|}
\cline{1-4}
\texttt{disksim\_disk} & \texttt{Read disconnect afterwrite} & float & required \\ 
\cline{1-4}
\multicolumn{4}{|p{6in}|}{
This specifies the processing time for a read request that disconnects
from the bus when the previous request was a write request. This
command processing overhead is applied after the disk determines that
disconnection is appropriate (prior to requesting disconnection from
the bus) and occurs in parallel with any ongoing media access.
}\\ 
\cline{1-4}
\multicolumn{4}{p{5in}}{}\\
\end{tabular}\\ 
\noindent 
\begin{tabular}{|p{\lpmodwidth}|p{\lpnamewidth}|p{0.5in}|p{0.5in}|}
\cline{1-4}
\texttt{disksim\_disk} & \texttt{Write disconnect overhead} & float & required \\ 
\cline{1-4}
\multicolumn{4}{|p{6in}|}{
This specifies the processing time for a write request that disconnects
from the bus (which generally occurs after the data are transferred
from the host to the on-board buffer/cache). This command processing
overhead is applied after the disk determines that disconnection is
appropriate (prior to requesting disconnection from the bus) and
occurs in parallel with any ongoing media access.
}\\ 
\cline{1-4}
\multicolumn{4}{p{5in}}{}\\
\end{tabular}\\ 
\noindent 
\begin{tabular}{|p{\lpmodwidth}|p{\lpnamewidth}|p{0.5in}|p{0.5in}|}
\cline{1-4}
\texttt{disksim\_disk} & \texttt{Extra write disconnect} & int & required \\ 
\cline{1-4}
\multicolumn{4}{|p{6in}|}{
This specifies whether or not the disk disconnects from the bus after
processing the write command but before any data have been transferred
over the bus into the disk buffer/cache. Although there are no
performance or reliability advantages to this behavior, it has been
observed in at least one production SCSI disk and has therefore been
included in DiskSim. If true~(1), the next five parameters configure
additional overhead values specifically related to this behavior.
}\\ 
\cline{1-4}
\multicolumn{4}{p{5in}}{}\\
\end{tabular}\\ 
\noindent 
\begin{tabular}{|p{\lpmodwidth}|p{\lpnamewidth}|p{0.5in}|p{0.5in}|}
\cline{1-4}
\texttt{disksim\_disk} & \texttt{Extradisc command overhead} & float & required \\ 
\cline{1-4}
\multicolumn{4}{|p{6in}|}{
This specifies the processing time for a write request that disconnects
from the bus before transferring any data to the disk buffer/cache.
This overhead is applied before requesting disconnection from the bus
and before any mechanical positioning delays. This parameter (when
enabled) functions in place of the above ``Write over.'' parameters.
}\\ 
\cline{1-4}
\multicolumn{4}{p{5in}}{}\\
\end{tabular}\\ 
\noindent 
\begin{tabular}{|p{\lpmodwidth}|p{\lpnamewidth}|p{0.5in}|p{0.5in}|}
\cline{1-4}
\texttt{disksim\_disk} & \texttt{Extradisc disconnect overhead} & float & required \\ 
\cline{1-4}
\multicolumn{4}{|p{6in}|}{
This specifies the additional processing time for a write request that
disconnects from the bus before transferring any data to the disk
buffer/cache. This overhead is also applied before requesting
disconnection from the bus, but it occurs in parallel with any
mechanical positioning delays. This parameter (when enabled)
functions in place of the above ``Write disconnect'' parameter for
initial write disconnections.
}\\ 
\cline{1-4}
\multicolumn{4}{p{5in}}{}\\
\end{tabular}\\ 
\noindent 
\begin{tabular}{|p{\lpmodwidth}|p{\lpnamewidth}|p{0.5in}|p{0.5in}|}
\cline{1-4}
\texttt{disksim\_disk} & \texttt{Extradisc inter-disconnect delay} & float & required \\ 
\cline{1-4}
\multicolumn{4}{|p{6in}|}{
This specifies the time between the initial disconnect from the bus and
the subsequent reconnection attempt for a write request that
disconnects from the bus before transferring any data to the disk
buffer/cache. It occurs in parallel with any mechanical positioning
delays.
}\\ 
\cline{1-4}
\multicolumn{4}{p{5in}}{}\\
\end{tabular}\\ 
\noindent 
\begin{tabular}{|p{\lpmodwidth}|p{\lpnamewidth}|p{0.5in}|p{0.5in}|}
\cline{1-4}
\texttt{disksim\_disk} & \texttt{Extradisc 2nd disconnect overhead} & float & required \\ 
\cline{1-4}
\multicolumn{4}{|p{6in}|}{
This specifies the processing time for a write request that disconnects
from the bus after data has been transferred but previously had
disconnected without transferring any data to the disk buffer/cache.
This command processing overhead is applied after the disk determines
that disconnection is appropriate (prior to requesting disconnection
from the bus) and occurs in parallel with any ongoing media access.
This parameter (when enabled) functions in place of the above ``Write
disconnect'' parameter for non-initial write disconnections.
}\\ 
\cline{1-4}
\multicolumn{4}{p{5in}}{}\\
\end{tabular}\\ 
\noindent 
\begin{tabular}{|p{\lpmodwidth}|p{\lpnamewidth}|p{0.5in}|p{0.5in}|}
\cline{1-4}
\texttt{disksim\_disk} & \texttt{Extradisc seek delta} & float & required \\ 
\cline{1-4}
\multicolumn{4}{|p{6in}|}{
This specifies the additional delay between the completion of the initial
command processing overhead and the initiation of any mechanical
positioning for a write request that disconnects from the bus before
transferring any data to the disk buffer/cache. This delay occurs in
parallel with ongoing bus activity and related processing overheads.
}\\ 
\cline{1-4}
\multicolumn{4}{p{5in}}{}\\
\end{tabular}\\ 
\noindent 
\begin{tabular}{|p{\lpmodwidth}|p{\lpnamewidth}|p{0.5in}|p{0.5in}|}
\cline{1-4}
\texttt{disksim\_disk} & \texttt{Minimum seek delay} & float & required \\ 
\cline{1-4}
\multicolumn{4}{|p{6in}|}{
This specifies the minimum media access delay for a nonsequential write
request. That is, a nonsequential write request (after any command
processing overheads) must wait at least this amount of time before
accessing the disk media.
}\\ 
\cline{1-4}
\multicolumn{4}{p{5in}}{}\\
\end{tabular}\\ 
\noindent 
\begin{tabular}{|p{\lpmodwidth}|p{\lpnamewidth}|p{0.5in}|p{0.5in}|}
\cline{1-4}
\texttt{disksim\_disk} & \texttt{Immediate buffer read} & int & required \\ 
\cline{1-4}
\multicolumn{4}{|p{6in}|}{
This specifies whether or not disk sectors should be transferred into the
on-board buffer in the order that they pass under the read/write head
rather than in strictly ascending logical block order. This is known
as {\it zero-latency reads} or {\it read-on-arrival}. It is intended
to improve response times by reducing rotational latency (by not
rotating all the way around to the first requested sector before
beginning to fill the buffer/cache).
}\\ 
\cline{1-4}
\multicolumn{4}{p{5in}}{}\\
\end{tabular}\\ 
\noindent 
\begin{tabular}{|p{\lpmodwidth}|p{\lpnamewidth}|p{0.5in}|p{0.5in}|}
\cline{1-4}
\texttt{disksim\_disk} & \texttt{Immediate buffer write} & int & required \\ 
\cline{1-4}
\multicolumn{4}{|p{6in}|}{
This specifies whether or not disk sectors should be transferred from the
on-board buffer in the order that they pass under the read/write head
rather than in strictly ascending logical block order. These are known
as {\it zero-latency writes} or {\it write-on-arrival}. It is
intended to improve response times by reducing rotational latency (by
not rotating all the way around to the first ``dirty'' sector before
beginning to flush the buffer/cache).
}\\ 
\cline{1-4}
\multicolumn{4}{p{5in}}{}\\
\end{tabular}\\ 
