\noindent 
\begin{tabular}{|p{\lpmodwidth}|p{\lpnamewidth}|p{0.5in}|p{0.5in}|}
\cline{1-4}
\texttt{disksim\_logorg} & \texttt{Addressing mode} & string & required \\ 
\cline{1-4}
\multicolumn{4}{|p{6in}|}{
This specifies how the logical data organization is addressed.
\texttt{Array} indicates that there is a single logical device number for
the entire logical organization.
\texttt{Parts} indicates that back-end storage devices are addressed as
though there were no logical organization, and requests are re-mapped
appropriately.
}\\ 
\cline{1-4}
\multicolumn{4}{p{5in}}{}\\
\end{tabular}\\ 
\noindent 
\begin{tabular}{|p{\lpmodwidth}|p{\lpnamewidth}|p{0.5in}|p{0.5in}|}
\cline{1-4}
\texttt{disksim\_logorg} & \texttt{Distribution scheme} & string & required \\ 
\cline{1-4}
\multicolumn{4}{|p{6in}|}{
This specifies the data distribution scheme (which is orthogonal to
the redundancy scheme).
\texttt{Asis} indicates that no re-mapping occurs.
\texttt{Striped} indicates that data are striped over the organization
members.
\texttt{Random} indicates that a random disk is selected for each request.
N.B.: This is only to be used with constant access-time disks
for load-balancing experiments.
\texttt{Ideal} indicates that an idealized data distribution (from a load
balancing perspective) should be simulated by assigning requests to
disks in a round-robin fashion. Note that the last two schemes do not
model real data layouts. In particular, two requests to the same
block will often be sent to different devices. However, these data
distribution schemes are useful for investigating various load
balancing techniques \cite{Ganger93a}.
N.B.: This is only to be used with constant access-time disks
for load-balancing experiments.
}\\ 
\cline{1-4}
\multicolumn{4}{p{5in}}{}\\
\end{tabular}\\ 
\noindent 
\begin{tabular}{|p{\lpmodwidth}|p{\lpnamewidth}|p{0.5in}|p{0.5in}|}
\cline{1-4}
\texttt{disksim\_logorg} & \texttt{Redundancy scheme} & string & required \\ 
\cline{1-4}
\multicolumn{4}{|p{6in}|}{
This specifies the redundancy scheme (which is orthogonal to the data
distribution scheme).
\texttt{Noredun} indicates that no redundancy is employed.
\texttt{Shadowed} indicates that one or more replicas of each data disk are
maintained.
\texttt{Parity\_disk} indicates that one parity disk is maintained to
protect the data of the other organization members.
\texttt{Parity\_rotated} indicates that one disk's worth of data (spread
out across all disks) are dedicated to holding parity information that
protects the other N-1 disks' worth of data in an N-disk organization.
%``Parity\_declustered'' indicates that a complex parity-based scheme
%such as parity declustering \cite{Holland92} is used to protect the
%organization's data. The last option is not supported in the first
%release of DiskSim.
}\\ 
\cline{1-4}
\multicolumn{4}{p{5in}}{}\\
\end{tabular}\\ 
\noindent 
\begin{tabular}{|p{\lpmodwidth}|p{\lpnamewidth}|p{0.5in}|p{0.5in}|}
\cline{1-4}
\texttt{disksim\_logorg} & \texttt{Components} & string & optional \\ 
\cline{1-4}
\multicolumn{4}{|p{6in}|}{
This specifies whether the data organization's component members are entire
disks (\texttt{Whole}) or partial disks (\texttt{Partial}). Only the former
option is supported in the first released version of DiskSim.
}\\ 
\cline{1-4}
\multicolumn{4}{p{5in}}{}\\
\end{tabular}\\ 
\noindent 
\begin{tabular}{|p{\lpmodwidth}|p{\lpnamewidth}|p{0.5in}|p{0.5in}|}
\cline{1-4}
\texttt{disksim\_logorg} & \texttt{devices} & list & required \\ 
\cline{1-4}
\multicolumn{4}{|p{6in}|}{
List of device names to be included in this logical organization.
}\\ 
\cline{1-4}
\multicolumn{4}{p{5in}}{}\\
\end{tabular}\\ 
\noindent 
\begin{tabular}{|p{\lpmodwidth}|p{\lpnamewidth}|p{0.5in}|p{0.5in}|}
\cline{1-4}
\texttt{disksim\_logorg} & \texttt{Stripe unit} & int & required \\ 
\cline{1-4}
\multicolumn{4}{|p{6in}|}{
This specifies the stripe unit size. 0~indicates fine-grained striping
(e.g.,~bit or byte striping), wherein all data disks in the logical
organization contain an equal fraction of every addressable data unit.
}\\ 
\cline{1-4}
\multicolumn{4}{p{5in}}{}\\
\end{tabular}\\ 
\noindent 
\begin{tabular}{|p{\lpmodwidth}|p{\lpnamewidth}|p{0.5in}|p{0.5in}|}
\cline{1-4}
\texttt{disksim\_logorg} & \texttt{Synch writes for safety} & int & required \\ 
\cline{1-4}
\multicolumn{4}{|p{6in}|}{
This specifies whether or not an explicit effort should be made to do the
N+1 writes of a parity-protected logical organization at ``the same
time'' when handling a front-end write request with the
read-modify-write (RMW) approach to parity computation.
If true~(1), then all reading of old values (for computing updated
parity values) must be completed before the set of back-end writes is issued.
If false~(0), then each back-end write is issued immediately after the
corresponding read completes (perhaps offering improved performance).
}\\ 
\cline{1-4}
\multicolumn{4}{p{5in}}{}\\
\end{tabular}\\ 
\noindent 
\begin{tabular}{|p{\lpmodwidth}|p{\lpnamewidth}|p{0.5in}|p{0.5in}|}
\cline{1-4}
\texttt{disksim\_logorg} & \texttt{Number of copies} & int & required \\ 
\cline{1-4}
\multicolumn{4}{|p{6in}|}{
This specifies the number of copies of each data disk if the logical
organization employs \texttt{Shadowed} redundancy. Otherwise, this
parameter is ignored.
}\\ 
\cline{1-4}
\multicolumn{4}{p{5in}}{}\\
\end{tabular}\\ 
\noindent 
\begin{tabular}{|p{\lpmodwidth}|p{\lpnamewidth}|p{0.5in}|p{0.5in}|}
\cline{1-4}
\texttt{disksim\_logorg} & \texttt{Copy choice on read} & int & required \\ 
\cline{1-4}
\multicolumn{4}{|p{6in}|}{
This specifies the policy used for selecting which disk from a set of
\texttt{Shadowed} replicas should service a given read request since any
of them can potentially do so.
1~indicates that all read requests are sent to a single primary
replica.
2~indicates that one of the replicas should be randomly selected for
each read request.
3~indicates that requests should be assigned to replicas in a
round-robin fashion.
4~indicates that the replica that would incur the shortest seek
distance should be selected and ties are broken by random selection.
5~indicates that the replica that has the shortest request queue
should be selected and ties are broken by random selection.
6~indicates that the replica that has the shortest request queue
should be selected and ties are broken by policy~4 (see above). This
parameter is ignored if \texttt{Shadowed} replication is not chosen.
}\\ 
\cline{1-4}
\multicolumn{4}{p{5in}}{}\\
\end{tabular}\\ 
\noindent 
\begin{tabular}{|p{\lpmodwidth}|p{\lpnamewidth}|p{0.5in}|p{0.5in}|}
\cline{1-4}
\texttt{disksim\_logorg} & \texttt{RMW vs. reconstruct} & float & required \\ 
\cline{1-4}
\multicolumn{4}{|p{6in}|}{
This specifies the breakpoint in selecting Read-Modify-Write (RMW) parity
updates (verses complete reconstruction) as the fraction of data disks
that are updated. If the number of disks updated by the front-end
write request is smaller than the breakpoint, then the RMW of the
``old'' data, ``old'' parity, and ``new'' data is used to compute the
new parity. Otherwise, the unmodified data in the affected stripe are
read from the corresponding data disks and combined with the new data
to calculate the new parity. This parameter is ignored unless some
form of parity-based replication is chosen.
}\\ 
\cline{1-4}
\multicolumn{4}{p{5in}}{}\\
\end{tabular}\\ 
\noindent 
\begin{tabular}{|p{\lpmodwidth}|p{\lpnamewidth}|p{0.5in}|p{0.5in}|}
\cline{1-4}
\texttt{disksim\_logorg} & \texttt{Parity stripe unit} & int & required \\ 
\cline{1-4}
\multicolumn{4}{|p{6in}|}{
This specifies the stripe unit size used for the \texttt{Parity\_rotated}
redundancy scheme. This parameter is ignored for other schemes. The
parity stripe unit size does not have to be equal to the stripe unit
size, but one must be a multiple of the other. Use of non-equal
stripe unit sizes for data and parity has not been thoroughly tested
in the current release of DiskSim.
}\\ 
\cline{1-4}
\multicolumn{4}{p{5in}}{}\\
\end{tabular}\\ 
\noindent 
\begin{tabular}{|p{\lpmodwidth}|p{\lpnamewidth}|p{0.5in}|p{0.5in}|}
\cline{1-4}
\texttt{disksim\_logorg} & \texttt{Parity rotation type} & int & required \\ 
\cline{1-4}
\multicolumn{4}{|p{6in}|}{
This specifies how parity is rotated
among the disks of the logical organization. The four options, as
described in \cite{Lee91}, are
1 - left symmetric,
2 - left asymmetric,
3 - right asymmetric,
4 - right symmetric.
This parameter is ignored unless \texttt{Parity\_rotated} redundancy is
chosen.
}\\ 
\cline{1-4}
\multicolumn{4}{p{5in}}{}\\
\end{tabular}\\ 
\noindent 
\begin{tabular}{|p{\lpmodwidth}|p{\lpnamewidth}|p{0.5in}|p{0.5in}|}
\cline{1-4}
\texttt{disksim\_logorg} & \texttt{Time stamp interval} & float & optional \\ 
\cline{1-4}
\multicolumn{4}{|p{6in}|}{
This specifies the interval between ``time stamps.'' A value of $0.0$ for
this parameter disables the time stamp mechanism.
}\\ 
\cline{1-4}
\multicolumn{4}{p{5in}}{}\\
\end{tabular}\\ 
\noindent 
\begin{tabular}{|p{\lpmodwidth}|p{\lpnamewidth}|p{0.5in}|p{0.5in}|}
\cline{1-4}
\texttt{disksim\_logorg} & \texttt{Time stamp start time} & float & optional \\ 
\cline{1-4}
\multicolumn{4}{|p{6in}|}{
This specifies the simulated time (relative to the beginning of the
simulation) of the first time stamp.
}\\ 
\cline{1-4}
\multicolumn{4}{p{5in}}{}\\
\end{tabular}\\ 
\noindent 
\begin{tabular}{|p{\lpmodwidth}|p{\lpnamewidth}|p{0.5in}|p{0.5in}|}
\cline{1-4}
\texttt{disksim\_logorg} & \texttt{Time stamp stop time} & float & optional \\ 
\cline{1-4}
\multicolumn{4}{|p{6in}|}{
This specifies the simulated time (relative to the beginning of the
simulation) of the last time stamp.
}\\ 
\cline{1-4}
\multicolumn{4}{p{5in}}{}\\
\end{tabular}\\ 
\noindent 
\begin{tabular}{|p{\lpmodwidth}|p{\lpnamewidth}|p{0.5in}|p{0.5in}|}
\cline{1-4}
\texttt{disksim\_logorg} & \texttt{Time stamp file name} & string & optional \\ 
\cline{1-4}
\multicolumn{4}{|p{6in}|}{
This specifies the name of the output file to contain a log of the
instantaneous queue lengths of each of the organization's back-end
devices at each time stamp. Each line of the output file corresponds
to a single time stamp and contains the queue lengths of each device
separated by white space. A value of ``0'' or of ``null'' disables
this feature (as does disabling the time stamp mechanism).
}\\ 
\cline{1-4}
\multicolumn{4}{p{5in}}{}\\
\end{tabular}\\ 
